%% Generated by Sphinx.
\def\sphinxdocclass{report}
\documentclass[letterpaper,10pt,english]{sphinxmanual}
\ifdefined\pdfpxdimen
   \let\sphinxpxdimen\pdfpxdimen\else\newdimen\sphinxpxdimen
\fi \sphinxpxdimen=.75bp\relax

\PassOptionsToPackage{warn}{textcomp}
\usepackage[utf8]{inputenc}
\ifdefined\DeclareUnicodeCharacter
% support both utf8 and utf8x syntaxes
  \ifdefined\DeclareUnicodeCharacterAsOptional
    \def\sphinxDUC#1{\DeclareUnicodeCharacter{"#1}}
  \else
    \let\sphinxDUC\DeclareUnicodeCharacter
  \fi
  \sphinxDUC{00A0}{\nobreakspace}
  \sphinxDUC{2500}{\sphinxunichar{2500}}
  \sphinxDUC{2502}{\sphinxunichar{2502}}
  \sphinxDUC{2514}{\sphinxunichar{2514}}
  \sphinxDUC{251C}{\sphinxunichar{251C}}
  \sphinxDUC{2572}{\textbackslash}
\fi
\usepackage{cmap}
\usepackage[T1]{fontenc}
\usepackage{amsmath,amssymb,amstext}
\usepackage{babel}



\usepackage{times}
\expandafter\ifx\csname T@LGR\endcsname\relax
\else
% LGR was declared as font encoding
  \substitutefont{LGR}{\rmdefault}{cmr}
  \substitutefont{LGR}{\sfdefault}{cmss}
  \substitutefont{LGR}{\ttdefault}{cmtt}
\fi
\expandafter\ifx\csname T@X2\endcsname\relax
  \expandafter\ifx\csname T@T2A\endcsname\relax
  \else
  % T2A was declared as font encoding
    \substitutefont{T2A}{\rmdefault}{cmr}
    \substitutefont{T2A}{\sfdefault}{cmss}
    \substitutefont{T2A}{\ttdefault}{cmtt}
  \fi
\else
% X2 was declared as font encoding
  \substitutefont{X2}{\rmdefault}{cmr}
  \substitutefont{X2}{\sfdefault}{cmss}
  \substitutefont{X2}{\ttdefault}{cmtt}
\fi


\usepackage[Bjarne]{fncychap}
\usepackage{sphinx}

\fvset{fontsize=\small}
\usepackage{geometry}

% Include hyperref last.
\usepackage{hyperref}
% Fix anchor placement for figures with captions.
\usepackage{hypcap}% it must be loaded after hyperref.
% Set up styles of URL: it should be placed after hyperref.
\urlstyle{same}
\addto\captionsenglish{\renewcommand{\contentsname}{Contents:}}

\usepackage{sphinxmessages}
\setcounter{tocdepth}{1}



\title{uberSpark Documentation}
\date{Aug 30, 2019}
\release{Version: 5.0; Release Series: Chase}
\author{https://uberspark.org}
\newcommand{\sphinxlogo}{\vbox{}}
\renewcommand{\releasename}{Release}
\makeindex
\begin{document}

\pagestyle{empty}
\sphinxmaketitle
\pagestyle{plain}
\sphinxtableofcontents
\pagestyle{normal}
\phantomsection\label{\detokenize{index::doc}}


Described below are details on the software requirements and
dependencies, build, verification and intallation of the uberSpark core
libraries and hardware model


\chapter{Software Requirements and Dependencies}
\label{\detokenize{sw-requirements:software-requirements-and-dependencies}}\label{\detokenize{sw-requirements::doc}}
We assume your are working in: \sphinxcode{\sphinxupquote{/home/\textless{}home-dir\textgreater{}/\textless{}work-dir\textgreater{}}}

Replace \sphinxcode{\sphinxupquote{\textless{}home-dir\textgreater{}}} with your home-directory name and \sphinxcode{\sphinxupquote{\textless{}work-dir\textgreater{}}}
with any working directory of your choice.


\section{Development OS and Base Packages}
\label{\detokenize{sw-requirements:development-os-and-base-packages}}
You will need a working Ubuntu 16.04.x LTS 64-bit environment for development and
verification. This can either be a Virtual Machine (VM) (e.g., VirtualBox) or a
container (e.g., Windows WSL). As of this writing, the Ubuntu 16.04.x LTS VM ISO
image is available at:

\begin{sphinxVerbatim}[commandchars=\\\{\}]
\PYG{n}{http}\PYG{p}{:}\PYG{o}{/}\PYG{o}{/}\PYG{n}{releases}\PYG{o}{.}\PYG{n}{ubuntu}\PYG{o}{.}\PYG{n}{com}\PYG{o}{/}\PYG{l+m+mf}{16.04}\PYG{o}{/}\PYG{n}{ubuntu}\PYG{o}{\PYGZhy{}}\PYG{l+m+mf}{16.04}\PYG{o}{.}\PYG{l+m+mi}{6}\PYG{o}{\PYGZhy{}}\PYG{n}{desktop}\PYG{o}{\PYGZhy{}}\PYG{n}{amd64}\PYG{o}{.}\PYG{n}{iso}
\end{sphinxVerbatim}

You will need to first perform an \sphinxcode{\sphinxupquote{update}} to download the latest package
lists from the repositories as shown below:

\begin{sphinxVerbatim}[commandchars=\\\{\}]
\PYG{n}{sudo} \PYG{n}{apt}\PYG{o}{\PYGZhy{}}\PYG{n}{get} \PYG{n}{update}
\end{sphinxVerbatim}

After the update completes, you will need to install the following base
packages required for development as shown below:

\begin{sphinxVerbatim}[commandchars=\\\{\}]
\PYG{n}{sudo} \PYG{n}{apt}\PYG{o}{\PYGZhy{}}\PYG{n}{get} \PYG{n}{install} \PYG{n}{git} \PYG{n}{gcc} \PYG{n}{binutils} \PYG{n}{autoconf}
\PYG{n}{sudo} \PYG{n}{apt}\PYG{o}{\PYGZhy{}}\PYG{n}{get} \PYG{n}{install} \PYG{n}{lib32z1} \PYG{n}{lib32ncurses5} \PYG{n}{lib32bz2}\PYG{o}{\PYGZhy{}}\PYG{l+m+mf}{1.0} \PYG{n}{gcc}\PYG{o}{\PYGZhy{}}\PYG{n}{multilib}
\PYG{n}{sudo} \PYG{n}{apt}\PYG{o}{\PYGZhy{}}\PYG{n}{get} \PYG{n}{install} \PYG{n}{ocaml} \PYG{n}{ocaml}\PYG{o}{\PYGZhy{}}\PYG{n}{findlib} \PYG{n}{ocaml}\PYG{o}{\PYGZhy{}}\PYG{n}{native}\PYG{o}{\PYGZhy{}}\PYG{n}{compilers}
\PYG{n}{sudo} \PYG{n}{apt}\PYG{o}{\PYGZhy{}}\PYG{n}{get} \PYG{n}{install} \PYG{n}{graphviz} \PYG{n}{libzarith}\PYG{o}{\PYGZhy{}}\PYG{n}{ocaml}\PYG{o}{\PYGZhy{}}\PYG{n}{dev} \PYG{n}{libfindlib}\PYG{o}{\PYGZhy{}}\PYG{n}{ocaml}\PYG{o}{\PYGZhy{}}\PYG{n}{dev}
\PYG{n}{sudo} \PYG{n}{apt}\PYG{o}{\PYGZhy{}}\PYG{n}{get} \PYG{n}{install} \PYG{n}{make} \PYG{n}{unzip}
\end{sphinxVerbatim}


\section{OCaml Compiler and Base Packages}
\label{\detokenize{sw-requirements:ocaml-compiler-and-base-packages}}
You will then need to install the OCaml Package manager as shown below:

\begin{sphinxVerbatim}[commandchars=\\\{\}]
\PYG{n}{wget} \PYG{n}{https}\PYG{p}{:}\PYG{o}{/}\PYG{o}{/}\PYG{n}{raw}\PYG{o}{.}\PYG{n}{github}\PYG{o}{.}\PYG{n}{com}\PYG{o}{/}\PYG{n}{ocaml}\PYG{o}{/}\PYG{n}{opam}\PYG{o}{/}\PYG{n}{master}\PYG{o}{/}\PYG{n}{shell}\PYG{o}{/}\PYG{n}{opam}\PYGZbs{}\PYG{n}{\PYGZus{}installer}\PYG{o}{.}\PYG{n}{sh} \PYG{o}{\PYGZhy{}}\PYG{n}{O} \PYG{o}{\PYGZhy{}} \PYGZbs{}\PYG{o}{\textbar{}} \PYG{n}{sh} \PYG{o}{\PYGZhy{}}\PYG{n}{s} \PYG{o}{/}\PYG{n}{usr}\PYG{o}{/}\PYG{n}{local}\PYG{o}{/}\PYG{n+nb}{bin}
\end{sphinxVerbatim}

After the OCaml Package Manager installs successfully, configure the opam environment and switch to
the appropriate OCaml compiler version as shown below:

\begin{sphinxVerbatim}[commandchars=\\\{\}]
eval {}`{}`opam config env{}`{}`
opam switch 4.02.3
\end{sphinxVerbatim}

After the opam environment switch, install the following opam packages in order:

\begin{sphinxVerbatim}[commandchars=\\\{\}]
\PYG{n}{opam} \PYG{n}{install} \PYG{n}{menhir}\PYG{o}{.}\PYG{l+m+mi}{20170712}
\PYG{n}{opam} \PYG{n}{install} \PYG{n}{ocamlgraph}\PYG{o}{.}\PYG{l+m+mf}{1.8}\PYG{o}{.}\PYG{l+m+mi}{7}
\PYG{n}{opam} \PYG{n}{install} \PYG{n}{ocamlfind}\PYG{o}{.}\PYG{l+m+mf}{1.7}\PYG{o}{.}\PYG{l+m+mi}{3}
\PYG{n}{opam} \PYG{n}{install} \PYG{n}{zarith}
\PYG{n}{opam} \PYG{n}{install} \PYG{n}{yojson}
\end{sphinxVerbatim}


\section{Coq Proof Assistant}
\label{\detokenize{sw-requirements:coq-proof-assistant}}
The Coq Proof Assistant is a required package for both CompCert as well as Frama-C.
You need to install the Coq Proof Assistant via opam as shown below:

\begin{sphinxVerbatim}[commandchars=\\\{\}]
\PYG{n}{opam} \PYG{n}{install} \PYG{n}{coq}\PYG{o}{.}\PYG{l+m+mf}{8.6}\PYG{o}{.}\PYG{l+m+mi}{1}
\end{sphinxVerbatim}


\section{CompCert Certified Compiler}
\label{\detokenize{sw-requirements:compcert-certified-compiler}}
The CompCert compiler is used to compile the C code for verified uberobjects within
uberSpark. The Compcert version currently supported is v3.0.1 and can be installed
as shown below:

\begin{sphinxVerbatim}[commandchars=\\\{\}]
\PYG{n}{wget} \PYG{n}{http}\PYG{p}{:}\PYG{o}{/}\PYG{o}{/}\PYG{n}{compcert}\PYG{o}{.}\PYG{n}{inria}\PYG{o}{.}\PYG{n}{fr}\PYG{o}{/}\PYG{n}{release}\PYG{o}{/}\PYG{n}{compcert}\PYG{o}{\PYGZhy{}}\PYG{l+m+mf}{3.1}\PYG{o}{.}\PYG{n}{tgz}
\PYG{n}{tar} \PYG{o}{\PYGZhy{}}\PYG{n}{xvzf} \PYG{n}{compcert}\PYG{o}{\PYGZhy{}}\PYG{l+m+mf}{3.1}\PYG{o}{.}\PYG{n}{tgz}
\PYG{n}{cd} \PYG{n}{CompCert}\PYG{o}{\PYGZhy{}}\PYG{l+m+mf}{3.1}
\PYG{o}{.}\PYG{o}{/}\PYG{n}{configure} \PYG{n}{x86\PYGZus{}32}\PYG{o}{\PYGZhy{}}\PYG{n}{linux}
\PYG{n}{make} \PYG{n+nb}{all}
\PYG{n}{sudo} \PYG{n}{make} \PYG{n}{install}
\PYG{n}{cd} \PYG{o}{.}\PYG{o}{.}
\end{sphinxVerbatim}


\section{Frama-C Verification Framework}
\label{\detokenize{sw-requirements:frama-c-verification-framework}}
The Frama-C verification framework is used to discharge uberobject invariants and
properties within uberSpark. The Frama-C version currently supported is
\sphinxcode{\sphinxupquote{Phosphorus-20170501}} and can be installed as shown below:

\begin{sphinxVerbatim}[commandchars=\\\{\}]
\PYG{n}{wget} \PYG{n}{http}\PYG{p}{:}\PYG{o}{/}\PYG{o}{/}\PYG{n}{frama}\PYG{o}{\PYGZhy{}}\PYG{n}{c}\PYG{o}{.}\PYG{n}{com}\PYG{o}{/}\PYG{n}{download}\PYG{o}{/}\PYG{n}{frama}\PYG{o}{\PYGZhy{}}\PYG{n}{c}\PYG{o}{\PYGZhy{}}\PYG{n}{Phosphorus}\PYG{o}{\PYGZhy{}}\PYG{l+m+mf}{20170501.}\PYG{n}{tar}\PYG{o}{.}\PYG{n}{gz}
\PYG{n}{tar} \PYG{o}{\PYGZhy{}}\PYG{n}{xvzf} \PYG{n}{frama}\PYG{o}{\PYGZhy{}}\PYG{n}{c}\PYG{o}{\PYGZhy{}}\PYG{n}{Phosphorus}\PYG{o}{\PYGZhy{}}\PYG{l+m+mf}{20170501.}\PYG{n}{tar}\PYG{o}{.}\PYG{n}{gz}
\PYG{n}{cd} \PYG{n}{frama}\PYG{o}{\PYGZhy{}}\PYG{n}{c}\PYG{o}{\PYGZhy{}}\PYG{n}{Phosphorus}\PYG{o}{\PYGZhy{}}\PYG{l+m+mi}{20170501}
\PYG{o}{.}\PYG{o}{/}\PYG{n}{configure}
\PYG{n}{make}
\PYG{n}{sudo} \PYG{n}{make} \PYG{n}{install}
\PYG{n}{cd} \PYG{o}{.}\PYG{o}{.}
\end{sphinxVerbatim}

You will also need to install Frama-C backend theorem provers such as
CVC3, Alt-Ergo and Z3. The WP Frama-C plugin manual referenced below
contains a chapter on installing the theorem provers:

\begin{sphinxVerbatim}[commandchars=\\\{\}]
\PYG{n}{http}\PYG{p}{:}\PYG{o}{/}\PYG{o}{/}\PYG{n}{frama}\PYG{o}{\PYGZhy{}}\PYG{n}{c}\PYG{o}{.}\PYG{n}{com}\PYG{o}{/}\PYG{n}{download}\PYG{o}{/}\PYG{n}{wp}\PYG{o}{\PYGZhy{}}\PYG{n}{manual}\PYG{o}{\PYGZhy{}}\PYG{n}{Phosphorus}\PYG{o}{\PYGZhy{}}\PYG{l+m+mf}{20170501.}\PYG{n}{pdf}
\end{sphinxVerbatim}

Note that you will need to install the correct versions of Why3 and the
provers as described in the aforementioned Frama-C WP plugin manual.
For example,  Why3 version 0.87.3 and Alt-ergo version 1.30. This can be
done via opam as shown below in the context of Why3:

\begin{sphinxVerbatim}[commandchars=\\\{\}]
\PYG{n}{opam} \PYG{n}{install} \PYG{n}{why3}\PYG{o}{.}\PYG{l+m+mf}{0.87}\PYG{o}{.}\PYG{l+m+mi}{3}
\end{sphinxVerbatim}


\chapter{Building and Installing uberSpark}
\label{\detokenize{build-install:building-and-installing-uberspark}}\label{\detokenize{build-install::doc}}

\section{Building uberSpark Tools}
\label{\detokenize{build-install:building-uberspark-tools}}
You will need to build the uberSpark toolchain before any other tasks.
For this purpose, While in the top-level directory of the uberSpark repository,
switch directory to uberSpark sources:

\begin{sphinxVerbatim}[commandchars=\\\{\}]
\PYG{n}{cd} \PYG{n}{src}
\end{sphinxVerbatim}

Then prepare for the build as below:

\begin{sphinxVerbatim}[commandchars=\\\{\}]
\PYG{o}{.}\PYG{o}{/}\PYG{n}{bsconfigure}\PYG{o}{.}\PYG{n}{sh}
\PYG{o}{.}\PYG{o}{/}\PYG{n}{configure}
\end{sphinxVerbatim}

And finally, build the toolchain:

\begin{sphinxVerbatim}[commandchars=\\\{\}]
\PYG{n}{make}
\end{sphinxVerbatim}


\section{Installing uberSpark}
\label{\detokenize{build-install:installing-uberspark}}
Upon a successful build, you will need to install the uberSpark toolchain,
system headers and hardware-model related files. You can do this using the
following command (while in the same directory of uberSpark sources \sphinxcode{\sphinxupquote{src/}}):

\begin{sphinxVerbatim}[commandchars=\\\{\}]
\PYG{n}{sudo} \PYG{n}{make} \PYG{n}{install}
\end{sphinxVerbatim}


\chapter{Verifying, Building and Installing uberSpark Libraries}
\label{\detokenize{verify-build-install-libs:verifying-building-and-installing-uberspark-libraries}}\label{\detokenize{verify-build-install-libs::doc}}
While in the top-level directory of the uberSpark source-tree, perform
the following tasks in order:
\begin{itemize}
\item {} 
Switch directory to UberSpark libraries sources \{\% highlight bash \%\}
cd src/libs \{\% endhighlight \%\}

\item {} 
Prepare for build \{\% highlight bash \%\} ./bsconfigure.sh ./configure
\{\% endhighlight \%\}

\item {} 
Verify UberSpark libraries \{\% highlight bash \%\} make
verify-ubersparklibs \{\% endhighlight \%\}

\item {} 
Build UberSpark libraries \{\% highlight bash \%\} make
build-ubersparklibs \{\% endhighlight \%\}

\item {} 
Install UberSpark libraries \{\% highlight bash \%\} sudo make install \{\%
endhighlight \%\}

\end{itemize}




\chapter{Indices and tables}
\label{\detokenize{index:indices-and-tables}}\begin{itemize}
\item {} 
\DUrole{xref,std,std-ref}{genindex}

\item {} 
\DUrole{xref,std,std-ref}{modindex}

\item {} 
\DUrole{xref,std,std-ref}{search}

\end{itemize}



\renewcommand{\indexname}{Index}
\printindex
\end{document}