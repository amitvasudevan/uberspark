%% Generated by Sphinx.
\def\sphinxdocclass{report}
\documentclass[letterpaper,10pt,english]{sphinxmanual}
\ifdefined\pdfpxdimen
   \let\sphinxpxdimen\pdfpxdimen\else\newdimen\sphinxpxdimen
\fi \sphinxpxdimen=.75bp\relax

\PassOptionsToPackage{warn}{textcomp}
\usepackage[utf8]{inputenc}
\ifdefined\DeclareUnicodeCharacter
% support both utf8 and utf8x syntaxes
  \ifdefined\DeclareUnicodeCharacterAsOptional
    \def\sphinxDUC#1{\DeclareUnicodeCharacter{"#1}}
  \else
    \let\sphinxDUC\DeclareUnicodeCharacter
  \fi
  \sphinxDUC{00A0}{\nobreakspace}
  \sphinxDUC{2500}{\sphinxunichar{2500}}
  \sphinxDUC{2502}{\sphinxunichar{2502}}
  \sphinxDUC{2514}{\sphinxunichar{2514}}
  \sphinxDUC{251C}{\sphinxunichar{251C}}
  \sphinxDUC{2572}{\textbackslash}
\fi
\usepackage{cmap}
\usepackage[T1]{fontenc}
\usepackage{amsmath,amssymb,amstext}
\usepackage{babel}



\usepackage{times}
\expandafter\ifx\csname T@LGR\endcsname\relax
\else
% LGR was declared as font encoding
  \substitutefont{LGR}{\rmdefault}{cmr}
  \substitutefont{LGR}{\sfdefault}{cmss}
  \substitutefont{LGR}{\ttdefault}{cmtt}
\fi
\expandafter\ifx\csname T@X2\endcsname\relax
  \expandafter\ifx\csname T@T2A\endcsname\relax
  \else
  % T2A was declared as font encoding
    \substitutefont{T2A}{\rmdefault}{cmr}
    \substitutefont{T2A}{\sfdefault}{cmss}
    \substitutefont{T2A}{\ttdefault}{cmtt}
  \fi
\else
% X2 was declared as font encoding
  \substitutefont{X2}{\rmdefault}{cmr}
  \substitutefont{X2}{\sfdefault}{cmss}
  \substitutefont{X2}{\ttdefault}{cmtt}
\fi


\usepackage[Bjarne]{fncychap}
\usepackage{sphinx}

\fvset{fontsize=\small}
\usepackage{geometry}

% Include hyperref last.
\usepackage{hyperref}
% Fix anchor placement for figures with captions.
\usepackage{hypcap}% it must be loaded after hyperref.
% Set up styles of URL: it should be placed after hyperref.
\urlstyle{same}
\addto\captionsenglish{\renewcommand{\contentsname}{Contents:}}

\usepackage{sphinxmessages}
\setcounter{tocdepth}{1}



\title{uberSpark Documentation}
\date{Aug 30, 2019}
\release{Version: 5.0; Release Series: Chase}
\author{https://uberspark.org}
\newcommand{\sphinxlogo}{\vbox{}}
\renewcommand{\releasename}{Release}
\makeindex
\begin{document}

\pagestyle{empty}
\sphinxmaketitle
\pagestyle{plain}
\sphinxtableofcontents
\pagestyle{normal}
\phantomsection\label{\detokenize{index::doc}}


Described below are details on the software requirements and
dependencies, build, verification and intallation of the uberSpark core
libraries and hardware model


\chapter{Software Requirements and Dependencies}
\label{\detokenize{sw-requirements:software-requirements-and-dependencies}}\label{\detokenize{sw-requirements::doc}}
We assume your are working in: \sphinxcode{\sphinxupquote{/home/\textless{}home-dir\textgreater{}/\textless{}work-dir\textgreater{}}}

Replace \sphinxcode{\sphinxupquote{\textless{}home-dir\textgreater{}}} with your home-directory name and \sphinxcode{\sphinxupquote{\textless{}work-dir\textgreater{}}}
with any working directory of your choice.


\section{Base OS and packages}
\label{\detokenize{sw-requirements:base-os-and-packages}}
You will need a working Ubuntu 16.04.x LTS 64-bit environment for development and
verification. This can either be a Virtual Machine (VM) (e.g., VirtualBox) or a
container (e.g., Windows WSL). As of this writing, the Ubuntu 16.04.x LTS VM ISO
image is available at:

\begin{sphinxVerbatim}[commandchars=\\\{\}]
\PYG{n}{http}\PYG{p}{:}\PYG{o}{/}\PYG{o}{/}\PYG{n}{releases}\PYG{o}{.}\PYG{n}{ubuntu}\PYG{o}{.}\PYG{n}{com}\PYG{o}{/}\PYG{l+m+mf}{16.04}\PYG{o}{/}\PYG{n}{ubuntu}\PYG{o}{\PYGZhy{}}\PYG{l+m+mf}{16.04}\PYG{o}{.}\PYG{l+m+mi}{6}\PYG{o}{\PYGZhy{}}\PYG{n}{desktop}\PYG{o}{\PYGZhy{}}\PYG{n}{amd64}\PYG{o}{.}\PYG{n}{iso}
\end{sphinxVerbatim}

You will need to first perform an \sphinxcode{\sphinxupquote{update}} to download the latest package
lists from the repositories as shown below:

\begin{sphinxVerbatim}[commandchars=\\\{\}]
\PYG{n}{sudo} \PYG{n}{apt}\PYG{o}{\PYGZhy{}}\PYG{n}{get} \PYG{n}{update}
\end{sphinxVerbatim}

After the update completes, you will need to install the following base
packages required for development as shown below:

\begin{sphinxVerbatim}[commandchars=\\\{\}]
\PYG{n}{sudo} \PYG{n}{apt}\PYG{o}{\PYGZhy{}}\PYG{n}{get} \PYG{n}{install} \PYG{n}{git} \PYG{n}{gcc} \PYG{n}{binutils} \PYG{n}{autoconf}
\PYG{n}{sudo} \PYG{n}{apt}\PYG{o}{\PYGZhy{}}\PYG{n}{get} \PYG{n}{install} \PYG{n}{lib32z1} \PYG{n}{lib32ncurses5} \PYG{n}{lib32bz2}\PYG{o}{\PYGZhy{}}\PYG{l+m+mf}{1.0} \PYG{n}{gcc}\PYG{o}{\PYGZhy{}}\PYG{n}{multilib}
\PYG{n}{sudo} \PYG{n}{apt}\PYG{o}{\PYGZhy{}}\PYG{n}{get} \PYG{n}{install} \PYG{n}{ocaml} \PYG{n}{ocaml}\PYG{o}{\PYGZhy{}}\PYG{n}{findlib} \PYG{n}{ocaml}\PYG{o}{\PYGZhy{}}\PYG{n}{native}\PYG{o}{\PYGZhy{}}\PYG{n}{compilers}
\PYG{n}{sudo} \PYG{n}{apt}\PYG{o}{\PYGZhy{}}\PYG{n}{get} \PYG{n}{install} \PYG{n}{graphviz} \PYG{n}{libzarith}\PYG{o}{\PYGZhy{}}\PYG{n}{ocaml}\PYG{o}{\PYGZhy{}}\PYG{n}{dev} \PYG{n}{libfindlib}\PYG{o}{\PYGZhy{}}\PYG{n}{ocaml}\PYG{o}{\PYGZhy{}}\PYG{n}{dev}
\PYG{n}{sudo} \PYG{n}{apt}\PYG{o}{\PYGZhy{}}\PYG{n}{get} \PYG{n}{install} \PYG{n}{make} \PYG{n}{unzip}
\end{sphinxVerbatim}
\begin{itemize}
\item {} 
OPAM (OCaml Package Manager) \{\% highlight bash \%\} wget
\sphinxurl{https://raw.github.com/ocaml/opam/master/shell/opam\_installer.sh} -O
- \textbar{} sh -s /usr/local/bin eval \sphinxcode{\sphinxupquote{opam config env}} opam switch 4.02.3
\{\% endhighlight \%\}

\item {} 
Menhir Parser (20170712) \{\% highlight bash \%\} opam install
menhir.20170712 \{\% endhighlight \%\}

\item {} 
ocamlgraph (1.8.7) \{\% highlight bash \%\} opam install ocamlgraph.1.8.7
\{\% endhighlight \%\}

\item {} 
ocamlfind (1.7.3) \{\% highlight bash \%\} opam install ocamlfind.1.7.3
\{\% endhighlight \%\}

\item {} 
coq proof assistant (8.6.1) \{\% highlight bash \%\} opam install
coq.8.6.1 \{\% endhighlight \%\}

\item {} 
zarith \{\% highlight bash \%\} opam install zarith \{\% endhighlight \%\}

\item {} 
yojson \{\% highlight bash \%\} opam install yojson \{\% endhighlight \%\}

\item {} 
Compcert (3.0.1) \{\% highlight bash \%\} wget
\sphinxurl{http://compcert.inria.fr/release/compcert-3.1.tgz} tar -xvzf
compcert-3.1.tgz cd CompCert-3.1 ./configure x86\_32-linux make all
sudo make install cd .. \{\% endhighlight \%\}

\item {} 
Frama-C (version Phosphorus-20170501) \{\% highlight bash \%\} wget
\sphinxurl{http://frama-c.com/download/frama-c-Phosphorus-20170501.tar.gz} tar
-xvzf frama-c-Phosphorus-20170501.tar.gz cd
frama-c-Phosphorus-20170501 ./configure make sudo make install cd ..
\{\% endhighlight \%\}

\item {} 
Install CVC3, Alt-Ergo and Z3 as backend theorem provers. The WP
Frama-C plugin manual (available
\sphinxhref{http://frama-c.com/download/wp-manual-Phosphorus-20170501.pdf}{here})
contains a chapter on installing the theorem provers. Note that you
will need to install the correct versions of Why3 and the provers as
described in the aforementioned Frama-C WP plugin manual (e.g., Why3
0.87.3 and Alt-ergo 1.30). This can be done via opam (e.g.,
\sphinxcode{\sphinxupquote{opam install why3.0.87.3}}).

\end{itemize}




\chapter{Indices and tables}
\label{\detokenize{index:indices-and-tables}}\begin{itemize}
\item {} 
\DUrole{xref,std,std-ref}{genindex}

\item {} 
\DUrole{xref,std,std-ref}{modindex}

\item {} 
\DUrole{xref,std,std-ref}{search}

\end{itemize}



\renewcommand{\indexname}{Index}
\printindex
\end{document}